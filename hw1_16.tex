\documentclass[12pt,epsf]{article}
\def\R{\mbox{\rlap{I}\hskip .03in R}}
\newcommand{\ind}[1]{{\bf 1}_{\{#1\}}}
\newcommand{\pr}[1]{\mbox{Pr}\left[#1\right]}
\newcommand{\bm}[1]{{\mbox{\boldmath $#1$}}}

\def\jayline{\underline{\ \ \ \ \ \ \ \ \ \ \ \ \ }}
\usepackage{graphicx}
\usepackage{color}
\definecolor{darkblue}{rgb}{.0, .0,.6}
\definecolor{darkgreen}{rgb}{.0, .6,.3}
\def\blue#1{{\color{darkblue}{#1}}}
\def\green#1{{\color{darkgreen}{#1}}}
%%%%%%%%%%%%%%%%%%%%%%%%%%%%%%%%%%%%%%%%%%%
\newcommand{\sol}[1]{%comment out next line to turn off solution
\textcolor{blue}{[Solution: #1]}
}%end definition of sol


\begin{document}
\noindent{\bf STAA552 Homework 1}

Due Friday, October 28, 2016.
%{\em All parts have equal weight.}




\begin{enumerate}
%\item Install the {\tt itsmr} package within {\tt R} on your computer.

\item Read Sections~1.1, 1.2, 1.3, 1.4 (you can skip 1.4.4 if you like), 1.5.1--1.5.5 and 2.1 of Agresti.

\item This is a HUGE EDIT.
\item Complete exercises 1.2, 2.1, 2.2a--b, and 2.3 of Agresti.

\item One version of the {\em prosecutor's fallacy} occurs when the probability of some evidence given that a defendant is not guilty is presented as if it were the probability that the defendant is not guilty given the evidence.  For example, let $M$ be the event that DNA known to have been left by the perpetrator at a crime scene matches the DNA of the defendant. Let $G$ denote the event that the defendant is guilty and $\bar G$ the event that the defendant is innocent.  Suppose that P$(M\mid\bar G)=1\times 10^{-6})$ (that is, the DNA could match by chance even if the defendant is not guilty).  Because the defendant's DNA matches the crime scene DNA, the prosecutor claims that there is a one in a million chance that the defendant is not guilty.  Suppose that the weight of all other evidence leads to P$(G)=1\times 10^{-4}$ (for example, other evidence narrows the set of suspects to 10,000 individuals, one of whom is the defendant).  Use the provided information to compute P$(\bar G\mid M)$.  Comment on your result and the prosecutor's claim.


 \item Use the data of Example 1.5.4 in Agresti.  Assume that the count of yellow seeds among the $n=8023$ hybrid seeds in Mendel's experiment follows a Binomial($n,\pi$) distribution.
     \begin{enumerate}
     \item Plot the log-likelihood function of the unknown parameter $\pi$ given the data, for a fine grid of values in the interval $(0,1)$.
     \item Zoom in by plotting the log-likelihood function of the unknown parameter $\pi$ given the data, for a fine grid of values in the interval $(0.65,0.85)$.  Large-sample normal approximations rely on the approximate quadratic shape of the log-likelihood in a neighborhood of the true value.  Does this quadratic approximation appear plausible here?
     \item Add a vertical line to your plot to indicate the location of $\widehat\pi$, the maximum likelihood estimator of $\pi$.

     \item Use equation (1.11) of Agresti to compute the score statistic for testing the null hypothesis $H_0:\pi=0.75$ versus the alternative $H_a:\pi\ne 0.75$. Compute the $p$-value of your test statistic using the normal approximation, and compute the $p$-value  of your {\em squared} test statistic using the $\chi^2_1$ approximation.  Interpret your results.\label{score}

     \item Use equation (1.16) of Agresti to compute Pearson's chi-squared statistic $X^2$ for Mendel's data.  Compute the $p$-value of the test statistic using the $\chi^2_1$ approximation, and compare to the results of the previous problem (\ref{score}).


     %\item Compute the Wald test statistic for the null hypothesis $H_0:\pi=0.5$ versus the alternative $H_a:\pi\ne 0.5$.
%     \item Compute the score test statistic for the null hypothesis $H_0:\pi=0.5$ versus the alternative $H_a:\pi\ne 0.5$.  Compute the $p$-value of your test statistic using the normal approximation, and compute the $p$-value  of your {\em squared} test statistic using the $\chi^2_1$ approximation.  Interpret your results.
%       \item Compute the likelihood ratio test statistic for the null hypothesis $H_0:\pi=0.5$ versus the alternative $H_a:\pi\ne 0.5$.  Compute the $p$-value of your test statistic  using the $\chi^2_1$ approximation.  Interpret your results.
     \end{enumerate}

 %\item Complete exercise 1.6 of Agresti.

\item Complete exercise 1.10 of Agresti.  Use five categories: 0 deaths, 1 death, 2 deaths, 3 deaths, or $\ge 4$ deaths, and note that you must estimate one parameter (see Agresti \S1.5.5).

\end{enumerate}


\enddocument
